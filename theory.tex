\section{Задача синтеза управлений при неопределенности}

Рассмотрим систему
\begin{equation}\label{problem}
    \dot{x} = A(t)x + B(t)u + C(t)v(t)
\end{equation}
с непрерывными матрицами \( A(t), B(t), C(t) \). Где \( x \in \mathbb{R}^n \)
 --- \emph{вектор состояния} системы; \( u \in \mathbb{R}^p \) --- 
 \emph{управление}, \( v \in \mathbb{R}^q \) --- \emph{внешнее возмущение},
 стесненные почти всюду по \( t \) некоторыми "геометрическими" ограничениями
\[
    u(t) \in \mathcal{P}(t), \quad v(t) \in \mathcal{Q}(t),
\]
 где \( \mathcal{P}(t) \) и \( \mathcal{Q}(t) \) --- заданные многозначные
 функции с выпуклыми компактными значениями, непрерывно зависящие от времени. 
Управление может быть выбрано в одном из двух классов:
\begin{itemize}
    \item в классе \( U \) программных управлений u = u(t) --- измеримых по
     Лебегу функций со значениями в \( \mathcal{P}(t) \) почти всюду.
    \item в классе \( U_\mathcal{P} \) позиционных управлений, представляющих
     собой многозначные функции \( \mathcal{U}(t,x) \subseteq
     \mathcal{P}(t) \). При этом выполнены условия существования и
     продолжаемости решения дифференциального включения
    \begin{equation}\label{dif_inclusion}
        \dot{x} \in A(t)x + B(t)\mathcal{U}(t,x) + C(t)v(t), \quad
         t_0 \le t \le t_1, 
    \end{equation}
    для любой измеримой по Лебегу функции \( v(t) \). \footnote{Примером класса
     \( U_\mathcal{P} \) может служить класс всех непрерывных по \( t \) и
     полунепрерывных сверху по \( x \) многозначных отображений с выпуклыми
     компактными значениями. В этом случае дифференциальное включение имеет
     решение на всем отрезке времени для произвольного \( x^0 = x(t_0) \),
     то есть существует абсолютно непрерывная функция \( x(t) \),
     удовлетворяющая дифференциальному включению почти всюду.[ссылка на
     доказательство леммы филлипова(например)]} В этой работе речь пойдет 
     именно про этот тип управлений.
\end{itemize}
Важно отметить, что в задачах с неопределенностью эти два типа управлений
 существенно не взимозаменямы. Имея позиционное управление, подстановкой его
 и решением задачи относительно \( u(t) \), уже нельзя однозначно найти
 программный аналог.

Пусть задано "целевое" множество \( \mathcal{M} \in \text{comp} \, \mathbb{R}^n \).
Задача \emph{о синтезе управления при неопределенности} состоит в отыскании
 \emph{множества разрешимости} \( \mathcal{W}(\tau, t_1, \mathcal{M}) = 
 \mathcal{W}[\tau] \) и \emph{позиционной стратегией управления}
 \( \mathcal{U}(t,x) \in U_{\mathcal{P}} \) таких, что все решения
 \eqref{dif_inclusion}, выпущенные из любой начальной позиции \( \{\tau, 
 x_{\tau}\}, \, x_{\tau} = x(\tau), \, x_{\tau} \in \mathcal{W}(\tau, t_1, 
 \mathcal{M}), \, \tau \in [t_0, t_1) \), достигали бы \emph{целевого множества}
 \( \mathcal{M} \) в момент времени \( t_1 \) при любом внешнем возмущени
 \( v(t) \in \mathcal{Q}(t) \). 

Задача имеет смысл в случае \( \mathcal{W}(\tau, t_1, \mathcal{M}) \ne 
 \varnothing \). Многозначная функция \( \mathcal{W}[t] = \mathcal{W}(t, 
 t_1,\mathcal{M}) \) называется \emph{трубкой разрешимости} или 
 \emph{мостом Красовского}, и является ключевым элементом в решении задачи.
 Существенным обстоятельством является возможность вычислить эту функцию 
 при помощи некоторого многозначного инетграла --- \emph{альтернированного 
 интеграла Л.С. Понтрягина}.

\section{Альтернированный интеграл Понтрягина}
Напомним определение этого интеграла. Для этого приведем вначале систему 
 \eqref{problem} к более простому виду
\begin{equation}\label{simplified_problem}
    \dot{x} = u + v
\end{equation}
 с новыми ограничениями
\begin{equation}\label{simplified_bounds}
    u \in \mathcal{P}_0(t), \quad v \in \mathcal{Q}_0(t),
\end{equation}
 где \( \mathcal{P}_0(t) = G(t_1, t) B(t) \mathcal{P}(t), \, \mathcal{Q}_0 = 
 G(t_1,t)C(t)\mathcal{Q}(t) \) и \( G(t,t_1) \) --- фундаментальная матрица
 однородного уравнения \eqref{problem}. Для этого сделаем \emph{невырожденную замену} 
 \( x(t) = G(t_1, t)x(t) \). Далее вместо \eqref{problem} будем
 рассматривать \eqref{simplified_problem} с ограничениями \eqref{simplified_bounds}, 
 опуская индекс нуль.
\begin{definition}
    Множеством разрешимости максиминного типа назовем множество
    \begin{equation}\label{maxmin_solvability_set}
        W[\tau] = W(\tau, t_1, \mathcal{M}) = \left\{ x: \, \max_v \min_u
         d^2(x(t_1), \mathcal{M}) \le 0 \mid x(\tau) = x \right\}, 
    \end{equation}
     где \( d^2(x. \mathcal{M}) = \min\limits_z \{ (x-z,x-z) \mid z \in 
     \mathcal{M} \} \) и \( x(t_1) \) --- конец в момент \( t_1 \) траектории
     \( x(t) \) системы \eqref{simplified_problem}, выпущенной из положения 
     \( x(\tau) = x \).
\end{definition}
\begin{statement}
    Для \( W[t] \) справедливо представление:
    \begin{equation}\label{solvability_set_format}
        W(t,t_1,\mathcal{M}) = \left( \mathcal{M} + \int\limits_t^{t_1} 
         (-\mathcal{P}(t)) dt \right) \dot{-} \int\limits_t^{t_1} \mathcal{Q}(t) dt,
         \quad \tau \le t \le t_1.
    \end{equation}
    Здесь символ \( \mathcal{P} \dot{-} \mathcal{Q} \) означает геометрическую(по 
     Минковскому) разность двух множеств \( \mathcal{P}, \, \mathcal{Q}\). А именно,
     \( c \in \mathcal{P} \dot{-} \mathcal{Q} \) тогда и только тогда, когда \(c +
     \mathcal{Q} \subseteq \mathcal{P} \).
\end{statement}
\begin{proof}
    При заданных \( u \) и \( v \) значение на правом конце выражается как 
    \[
        x(t_1) = x(t) + \int\limits_t^{t_1}u(\tau)d\tau + \int\limits_t^{t_1}
        v(\tau)d\tau.
    \]
    В таком случае множество разрешимости очевидно представляется в виде
    \begin{multline*}
        W(t, t_1, \mathcal{M}) = \bigcap\limits_{v \in \mathcal{Q}} 
         \bigcup\limits_{u \in \mathcal{P}} \left\{ x : x = x(t_1)
         -\int\limits_t^{t_1}u(\tau) d\tau - \int\limits_t^{t_1} v(\tau)
         d\tau, \, x(t_1) \in \mathcal{M} \right\} = \\
        = \bigcap\limits_{v \in \mathcal{Q}} \bigcup\limits_{u \in 
         \mathcal{P}} \left\{ \mathcal{M} - \int\limits_t^{t_1} u(\tau) d\tau
         - \int\limits_t^{t_1} v(\tau) d\tau \right\} = \\
        = \bigcap\limits_{v \in \mathcal{Q}} \left\{ \mathcal{M} +
         \int\limits_t^{t_1}(-\mathcal{P}(\tau)) d\tau - \int\limits_t^{t_1}
         v(\tau) d\tau \right\} = \\
        = \left\{ x : x \in \left( \mathcal{M} + \int\limits_t^{t_1}(- \mathcal{P}
         (\tau)) \right)  - \int\limits_t^{t_1} v(\tau) d\tau, \, \forall v(\tau)
         \in \mathcal{Q}(\tau) \right\} = \\
        = \left\{ x : x + \int\limits_t^{t_1} v(\tau) d\tau \in \mathcal{M} + 
         \int\limits_t^{t_1}(- \mathcal{P}(\tau)) d\tau, \, \forall v(\tau) \in
         \mathcal{Q}(\tau) \right\}.
    \end{multline*}
    Последнее в силу определения разности по Минковскому совпадает с правой частью
     формулы \eqref{solvability_set_format}.
\end{proof}

Опишем формальную процедуру построения альтернированного интеграла. Для этого
 построим множество \( \mathcal{W}^* (\tau, t_1, \mathcal{M}) \), явялющееся
 суперпозицей множеств \( W(\tau, t_1, \mathcal{M}) \), определенных выше. Взяв
 интервал \( \tau \le t \le t_1 \), рассмотрим разбиение \( \Sigma_k = \{ \sigma_1,
 \dots, \sigma_k \} \),
\[
    t = t_1 - \sum_{i=1}^k \sigma_i, \dots, t_1 - \sigma_1, t_1, \quad \sigma_i > 0.
\]
На первом шаге, начав с момента \( t_1 \), найдем множество \( W[t_1 - \sigma_1] = 
 W(t_1 - \sigma_1, t_1, \mathcal{M}) \). Вследствие \eqref{solvability_set_format}
 будем иметь
\begin{equation}\label{first_iter}
     W(t_1 - \sigma_1, t_1, \mathcal{M}) = \left( \mathcal{M} + \int\limits_{t_1 - \sigma_1}
     ^{t_1}(-\mathcal{P}(\tau)) d\tau \right) \dot{-} \int\limits_{t_1 - \sigma_1}^{t_1}
     \mathcal{Q}(\tau)d\tau.
\end{equation}
Продолжая последовательную процедуру, имеем
\begin{equation}
    \begin{gathered}\label{second_iter}
        W(t_1 - \sigma_1 -\sigma_2, t_1 - \sigma_1, W[t_1 - \sigma_1]) = \left( W[t_1 - \sigma_1]
        + \int\limits_{t_1 - \sigma_1 - \sigma_2}^{t_1 - \sigma_1} (-\mathcal{P}(\tau))d\tau \right)
        \dot{-} \\
        \dot{-} \int\limits_{t_1 - \sigma_1 - \sigma_2}^{t_1 - \sigma_1} \mathcal{Q}(\tau) d\tau
    \end{gathered}
\end{equation}
и, в итоге, получаем
\begin{equation}\label{pontr_iter}
    W(t, t + \sigma_k, W(t + \sigma_k, t + \sigma_k + \sigma_{k-1}, \dots, W(t_1 - \sigma_1, t_1,
    \mathcal{M}))) = \mathcal{I}(t, t_1, \mathcal{M}, \Sigma_k).
\end{equation}
В приведенной процедуре предполагается, что все возникающие множества \( W(\cdot) \) вида 
 \eqref{pontr_iter} непусты.
\begin{assumption}\label{assumption_nonempty}
    Существует такая непрерывная функция \( \beta(t) > 0, \, t \in [t_0, t_1] \), что все
     множества
    \begin{equation}
        \begin{gathered}
            W \left( t_1 - \sum_{i=1}^j\sigma_i, t_1 - \sum_{i = 1}^{j - 1}\sigma_i, W 
             \left( t_1 - \sum_{i = 1}^ {j - 1} \sigma_1, t_1 - \sum_{i = 1}^{j - 2} 
             \sigma_i, \dots, W(t_1 - \sigma_1, t_1, \mathcal{M}) \right) \dots \right)
             \dot{-} \\
            \dot{-} \ \beta \left( t_1 - \sum_{i = 1}^j \sigma_i
             \right) S \ne \varnothing 
        \end{gathered}
    \end{equation}
    при \( j = 1, \dots, k \), каким бы ни было разбиение \( \Sigma_k \).
\end{assumption}
Предположение \eqref{assumption_nonempty} обеспечивает условие \( W(\tau, t_1, \mathcal{M}) = 
 \mathcal{I}(t, t_1, \mathcal{M}, \Sigma_k) \ne \varnothing \) для любого разбиения \( \Sigma_k \).

Ниже всюду будем случитать предположение \eqref{assumption_nonempty} выполненым. Следует заметить,
 что данное замечание введено не только для облегчения выкеладок. В его отсутствие некоторые из
 утверждений, приводимых далее, могут оказаться неверными.

Следуя \eqref{first_iter} --- \eqref{pontr_iter}, приходим к аналитическому выражению
\begin{multline}\label{pontr_analytical}
    \mathcal{I}(\tau, t_1, \mathcal{M}, \Sigma_k) = \left( \dots \left( \left( \mathcal{M} + 
     \int\limits_{t_1 - \sigma_1}^{t_1} (-\mathcal{P}(\tau)) d\tau \right) \dot{-} \int\limits_
     {t_1 - \sigma_1}^{t_1} \mathcal{Q}(\tau) d\tau \right) + \dots \dot{-} \right. \\
    \left. \int\limits_t^{t + \sigma_k} \mathcal{Q}(\tau) d\tau \right).
\end{multline}

Множества \( \mathcal{I} (t, t_1, \mathcal{M}, \Sigma_k) \) есть выпуклые компакты для любого
 разбиения \( \Sigma_k \). Рассмотрим хаусдорфов предел этих множеств при \( \max \{\sigma_i : 
 \ i = 1, \dots, k \} \to 0 \).

 Напомним, что \emph{хаусдорфово полурасстояние} между компактами \( \mathcal{Q}, \mathcal{M} \)
 определяется как 
\[ 
    h_+(\mathcal{Q}, \mathcal{M}) = \max_x \min_z \{(x - z, x - z)^{1/2} \mid x \in \mathcal{Q}, 
     z \in \mathcal{M} \},
\]
в то время как \emph{хаусдорфово расстояние} \( h(\mathcal{Q}, \mathcal{M}) = \max
 \{ h_+(\mathcal{Q}, \mathcal{M}), h_+(\mathcal{M}, \mathcal{Q}) \}. \)

\begin{lemma}
    Существует хаусдорфов предел \( \mathcal{I}(t, t_1, \mathcal{M}) \)
    \[
        \lim h \left( \mathcal{I}(t, t_1, \mathcal{M}, \Sigma_k), \mathcal{I}(t, t_1, \mathcal{M})
         \right) = 0
    \]
    при 
    \[
        \max\{\sigma_i: i = 1,\dots, k \} \to 0, \quad k \to \infty, \quad \sum_{i = 1}^k \sigma_i 
         = t_1 - t. 
    \]
    Этот предел не зависит от способа разбиения \( \Sigma_k \).
\end{lemma}
Множество
\begin{equation}\label{alt_sovability_region}
    \mathcal{I}(t, t_1, \mathcal{M}) = \mathcal{W}^*(t,t_1,\mathcal{M}) = \mathcal{W}^*[t]
\end{equation}
будем называть \emph{альтернированной областью разрешимости} задачи \eqref{problem}, обозначая его 
 как 
\[
    \mathcal{I}(t, t_1, \mathcal{M}) = \int\limits_{t_1, \mathcal{M}}^t \left( (-\mathcal{P}(\tau))
     d\tau \dot{-} \mathcal{Q}(\tau) d\tau \right).
\]
Фактически множество \( \mathcal{I}(t, t_1, \mathcal{M}) \) есть значение некоторого многозначного
 интеграла, известного как \emph{альтернированный интеграл Понтрягина}. Этот интеграл детально
 описан в статье \cite{lin_dif_chasing}.

\begin{lemma}
    Многозначное отображение \( \mathcal{W}^*(t, t_1, \mathcal{M}) \) удовлетворяет полугрупповому
     свойству:
    \begin{equation}
        \mathcal{W}^*(t, t_1, \mathcal{M}) = \mathcal{W}^*(\tau, t, \mathcal{W}^*(t, t_1,\
         \mathcal{M})), \quad  \tau \le t \le t_1.
    \end{equation}
\end{lemma}
\begin{proof}
    Доказательство этого факта вытекает из свойства аддитивности альтернированного интеграла
     \( \mathcal{I}(t, t_1, \mathcal{M}) \), которое доказывается, например, в \cite{lin_dif_chasing}.
\end{proof}
Ключевое свойство альтернированного интеграла содержит следующая
\begin{theorem}
    Множество разрешимости \( \mathcal{W}[t] \) может быть представлено как 
    \begin{equation}
        \mathcal{W}[t] \equiv \mathcal{W}^*[t] \equiv \mathcal{I}(t, t_1, \mathcal{M}), \quad
         t_0 \le t \le t_1.
    \end{equation}
\end{theorem}

\section{Гарантированный синтез управлений. Функции цены}
Сформулированная задача \emph{гарантированного синтеза управлений при неопределенности} не содержит 
 каких-либо криетриев оптимальности. В ней требуется найти лишь некоторое допустимое "гарантированное"
 решение. Тем не менее будем рассматривать её посредством сведения к процедурам оптимизации, принятых
 в рамках идей динамического программирования.

Введем \emph{функцию цены} 
\begin{equation}
    \mathcal{V} = \min_{\mathcal{U}} \max_{x(\cdot)} \{\mathcal{I}(t,x) \mid \mathcal{U} \in 
     U_{\mathcal{P}}, \, x(\cdot) \in \mathcal{X}_{\mathcal{U}}(\cdot) \},
\end{equation}
где
\[
     \mathcal{I}(t,x) = d^2(x[t_1], \mathcal{M})
\]
и \( \mathcal{X}_{\mathcal{U}}(\cdot) \) --- множество всех траекторий \( x(\cdot) \) включения
\begin{equation}
    \dot{x} \in \mathcal{U}(t,x) + \mathcal{Q}(t), \quad x(\tau) = x,
\end{equation}
порожденных заданной стратегией \( \mathcal{U} \in U_{\mathcal{P}} \).

В таком случае для функции \( \mathcal{V}(t,x) \) можно составить следующее уравнение 
 Гамильтона-Якоби-Беллмана-Айзекса:
\begin{equation}\label{HJBI}
    \frac{\partial \mathcal{V}}{\partial t} + \min_u \max_v \left( \frac{\partial \mathcal{V}}
     {\partial x}, u + v \right) = 0, \quad u \in \mathcal{P}(t), \ v \in \mathcal{Q}(t),
\end{equation}
с граничным условием
\begin{equation}\label{HJBI_boundary}
    \mathcal{V}(t_1, x) = d^2(x, \mathcal{M}).
\end{equation}
В общем случае подобное уравнение может не иметь классического решения и для его рассмотрения
 следует привлекать понятия обобщенных "вязкостных" или "минимаксных" решений. В данной работе
 функция цены \( \mathcal{V}(t,x) \) оказывается выпуклой или квазивыпуклой по \( x \). Поэтому
 решения соответствующих уравнений Гамильтона-Якоби-Беллмана-Айзекса, рассматриваемых здесь, не
 выходят за рамки вязкостных или даже классических решений.

Перейдем к другой интерпретации функции цены \( \mathcal{V}(t,x) \). Рассмотрим интервал 
 \( \tau \le t \le t_1 \) и построим разбиение \( \Sigma_k = \{ \sigma_1, \dots, \sigma_k \} \),
 подобное рассмотренному в разделе 1. Для данного разбиения рассмотрим рекуррентные соотношения
\begin{equation*}
    V_k^+(t_1 - \sigma_1, x) = \left\{ \max_v \min_u d^2(x(t_1), \mathcal{M}) \mid t_1 -
     \sigma_1 \le t \le t_1, \ x(t_1 - \sigma_1) = x \right\},
\end{equation*}
\begin{multline*} 
    V_k^+(t_1 - \sigma_1 - \sigma_2, x) = \left\{ \max_v \min_u V_k^+(t_1 - \sigma_1, x(t_1 -
     \sigma_1)) \mid t_1 - \sigma_1 - \sigma_2 \le t \le t_1 - \sigma_1, \right. \\ 
    x(t_1 - \sigma_1 - \sigma_2) = x \Big\},
\end{multline*}
и так далее. Наконец в точке \( \tau = t_1 - \sum\limits_{i = 1}^k \sigma_i \):
\begin{equation*}
    V_k^+(\tau, x) = \left\{ \max_v \min_u V_k^+ (\tau + \sigma_k, x(\tau + \sigma_k)) \mid
     \tau \le t \le \tau + \sigma_k, \ x(\tau) = x \right\},
\end{equation*}
где \( v(t) \in \mathcal{Q}(t), u(t) \in  \mathcal{P}(t) \) почти всюду на соответствующих
 интервалах. Всюду предполагаем, что "максимины" существуют.

\begin{lemma}
    При
    \[ 
        \max_{i = 1,\dots,k} \{\sigma_i\}, \quad k \to \infty, \quad \sum_{i = 1}^k \sigma_i = 
         t_1 - \tau, 
    \]
    существует поточечный предел
    \[
        V^+(\tau, x) = \lim_{k \to \infty} V_k^+(\tau, x),
    \]
    не зависит от выбора разбиения \( \Sigma_k \).
\end{lemma}
Будем называть \( V^+(\tau, x) \) \emph{последовательным максимином}. Обозначим
\begin{equation}
    V^+(\tau, x) = V^+(\tau, x \mid V^+(t_1, \cdot)), \quad V^+(t_1, x) \equiv d^2(x,
     \mathcal{M}).
\end{equation}
Нетрудно заметить, что имеет место следующая
\begin{lemma}
    Функция \( V^+(\tau, x) \) удовлетворяет полурупповому свойству
    \[
        V^+(\tau, x \mid V^+(t_1, \cdot)) = V^+(\tau, x \mid V^+(t, \cdot \mid V^+(t_1, 
         \cdot))).  
    \]
    При этом справедливо неравенство
    \[
        V^+(t, x) \ge \left\{ \max_v \min_u V^+(t + \sigma, x(t + \sigma)) \mid x(t) = x 
         \right\}, \quad \sigma > 0.
    \]
\end{lemma}

Отметим связь функции с альтернированным интегралом.
\begin{lemma}
    Справедливо равенство
    \[
        V^+ (\tau, x) = d^2(x, \mathcal{W}^*(\tau, t_1, \mathcal{M})),
    \]
    где \( \mathcal{W}^*[\tau] = \mathcal{W}^*(\tau, t_1, \mathcal{M}) = \mathcal{I}(\tau,
     t_1, \mathcal{M}) \) --- значение альтернированного интеграла
\end{lemma}
Отюсда прямо следует
\begin{lemma}
    Альтернированный инетграл \( \mathcal{I}(\tau, t_1, \mathcal{M}) \) совпадает с
     множеством уровня последовательного максимина \( V^+(\tau, x) \):
    \[
        \mathcal{I}(\tau, t_1, \mathcal{M}) = \{ x : V^+(\tau, x) \le 0 \}.
    \]
\end{lemma}

Теперь рассмотрим задачу аналогичную предыдущей. Для произвольного \( \Sigma_k \), построим
 рекуррентные соотношения 
\begin{equation*}
    V_k^-(t_1 - \sigma_1, x) = \left\{ \min_u \max_v  d^2(x(t_1), \mathcal{M}) \mid t_1 -
     \sigma_1 \le t \le t_1, \ x(t_1 - \sigma_1) = x \right\},
\end{equation*}
\begin{multline*} 
    V_k^-(t_1 - \sigma_1 - \sigma_2, x) = \left\{ \min_u \max_v  V_k^-(t_1 - \sigma_1, x(t_1 -
     \sigma_1)) \mid t_1 - \sigma_1 - \sigma_2 \le t \le t_1 - \sigma_1, \right. \\ 
    x(t_1 - \sigma_1 - \sigma_2) = x \Big\},
\end{multline*}
и так далее. Наконец в точке \( \tau = t_1 - \sum\limits_{i = 1}^k \sigma_i \):
\begin{equation*}
    V_k^-(\tau, x) = \left\{ \min_u \max_v  V_k^- (\tau + \sigma_k, x(\tau + \sigma_k)) \mid
     \tau \le t \le \tau + \sigma_k, \ x(\tau) = x \right\},
\end{equation*}
Для определенных таким образом функций имеют место аналогичные утверждения
\begin{lemma}
    1. При
    \[ 
        \max_{i = 1,\dots,k} \{\sigma_i\}, \quad k \to \infty, \quad \sum_{i = 1}^k \sigma_i = 
         t_1 - \tau, 
    \]
    существует поточечный предел
    \[
        V^-(\tau, x) = \lim_{k \to \infty} V_k^-(\tau, x),
    \]
    не зависит от выбора разбиения \( \Sigma_k \). \\
    2. Функция \( V^-(\tau, x) \) удовлетворяет полугрупповому свойству.
    \begin{equation}
        V^-(\tau, x \mid V^-(t_1, \cdot)) = V^-(\tau, x \mid V^-(t, \cdot \mid V^-(t_1, \cdot))).
    \end{equation} \\
    3. Справедливо неравенство
    \begin{equation}
        V^-(t,x) \le \min_u \max_v \{ V^-(t + \sigma, x(t + \sigma)) \mid x(t) = x \}, \quad \sigma > 0.
    \end{equation}
\end{lemma}

Подобно тому как альтернированный интеграл \( I(\tau, t_1, \mathcal{M}) \) является множеством уровня
 последовательного максимина, множество уровня последовательного минимакса также представляет собой
 некоторый многозначный интеграл. Определим его, для чего введем следующее
\begin{definition}
    Множество разрешимости \( W^-[\tau] \) минимаксоного типа называется следующая совокупность векторов
    \[
        W_-[\tau] = W_-(\tau, t_1, \mathcal{M}) = \{ x : \min_u \max_v d^2(x(t_1), \mathcal{M}) \le 0 \mid
         x(\tau) = x \}.
    \]
\end{definition}
Для введенного множества справедливо представление аналогичное представлению 
\begin{statement}
    Для \( W_-[\tau] \) справедливо представление
    \begin{equation}
        W_-(\tau, t_1, \mathcal{M}) = \left( \mathcal{M} \dot{-} \int\limits_{\tau}^{t_1} \mathcal{Q}(t)
         dt \right) + \int\limits_{\tau}^{t_1} (-\mathcal{P}(t)) dt, \quad t_0 \le s \le t_1.
    \end{equation}
\end{statement}
Доказательство этого утверждения проводится аналогично доказательству соответствующего утверждения для
 множества максиминного типа.

Теперь, аналогично построению альтернированного интеграла Понтрягина \( \mathcal{I}(\tau, t_1, \mathcal{M}) \),
рассмотрим суперпозицию множеств \( W_-(t, t_1, \mathcal{M}) \) для произвольного разбиения \( \Sigma_k
 = \{ \sigma_1, \dots, \sigma_k\} \) отрезка \( \tau \le t \le t_1 \).

\[
    W_-(t_1 - \sigma_1, t_1, \mathcal{M}) = \left( \mathcal{M} \dot{-} \int\limits_{t_1 - \sigma_1}^{t_1}
     \mathcal{Q}(s) ds \right) + \int\limits_{t_1 - \sigma_1}^{t_1} (-\mathcal{P}(s))ds.
\]
И далее приходим к выражению
\begin{multline}\label{}
    \mathcal{I}_-(\tau, t_1, \mathcal{M}, \Sigma_k) = \left(\dots\left(\left(\mathcal{M} \dot{-} \int\limits_{t_1
     - \sigma_1}^{t_1} \mathcal{Q}(t) dt \right) + \int\limits_{t_1 - \sigma_1}^{t_1}(-\mathcal{P}(t)) dt \right) 
     \dot{-} \dots \right. \\
    \left. + \int\limits_{\tau}^{\tau + \sigma_k} (-\mathcal{P}(t)) dt \right) .
\end{multline}

Формальная процедура, описанная выше, предполгает что все множества \( \mathcal{I}_-[\cdot]\) не пусты.
Для этого необходимо принять
\begin{assumption}
    Существует непрерывная функция \( g(t) : [t_0, t_1] \to \mathbb{R}^n \) и число \( r \) такие, что для любого
     разбиения \( \Sigma_k \) справедливо включение
    \[
        g \left(t_1 - \sum_{i = 1}^j \sigma_i \right) + rB \subseteq \mathcal{I}_-\left[ t_1 - \sum_{i = 1}^j
         \sigma_i \right], \quad j = 1, \dots, m.
    \]
\end{assumption}

\begin{lemma}
    Существует замкнутое множество --- хаусдорфов предел \( \mathcal{I}_-(\tau, t_1, \mathcal{M}) \)
    \[
        \lim h \left( \mathcal{I}_-(\tau, t_1, \mathcal{M}, \Sigma_k), \mathcal{I}_-(\tau, t_1, \mathcal{M}) \right) = 0    
    \]
    при
    \[
        \max\{\sigma_i : i = 1, \dots, k \} \to 0, \quad k \to \infty, \quad \sum_{i = 1}^k \sigma_i = t_1 - \tau,
    \]
    и этот предел не зависит от выбора разбиений \( \Sigma_k \)
\end{lemma}

Будем называть
\[ 
    \mathcal{I}_-(\tau, t_1, \mathcal{M}) = \mathcal{W}_-(t, t_1, \mathcal{M}) = \mathcal{W}_-[t]
\]
\emph{альтернированным интегралом Л.С. Понтрягина второго рода}.

\begin{lemma}
    Справедливы соотношения:
    \[
        V^-(\tau, x) = d^2(x, \mathcal{W}_-(\tau, t_1, \mathcal{M})),
        \mathcal{I}(\tau, t_1, \mathcal{M}) = \left\{ x : V^-(\tau, x) \le 0 \right\} 
    \]
\end{lemma}

\begin{lemma}
    Многозначное отображение \( \mathcal{W}^-(t, t_1, \mathcal{M}) \) удовлетворяет полугрупповому свойству, а именно
    \[
        \mathcal{W}_-(\tau, t_1, \mathcal{M}) = \mathcal{W}_-(\tau, t, \mathcal{W}_-(t, t_1, \mathcal{M})), \quad \tau
         \le t \le t_1.
    \]

\end{lemma}

Отметим, что алтернированные интегралы соответствуют пределам разных информационных схем. Так, альтернированный интеграл
 Понтрягина предполагает появление текущей информации о помехах с опережением, в то время как интеграл второго рода ---
 с запаздываанием. Тем не менее, в пределе эти схемы дают один и тот же результат, а именно, справедливо утверждение.

\begin{theorem}
    Альтернированный интеграл Понтрягина совпадает с альтернированным интегралом второго рода:
    \[
        \mathcal{W}(t, t_1, \mathcal{M}) = \mathcal{W}_-(t, t_1, \mathcal{M}).
    \]
\end{theorem}

\section{Решение задачи синтеза}
Суммируем основыне утверждения в виде следуещей теоремы.
\begin{theorem}
    1. Последовательные максимин \( V^+(\tau, x) \) и минимакс \( V^-\) равны и совпадают с функцией цены \( \mathcal{V}(\tau,x) \):
    \[
        V^+(\tau, x) \equiv V^-(\tau, x) \equiv \mathcal{V}(\tau, x),
    \]
    2. Множество разрешимости задачи допускает следющие эквивалентные представления:
    \[
        \mathcal{I}(\tau, t_1, \mathcal{M}) = \mathcal{W}^*[\tau] = \{ x : \mathcal{V}(\tau \le 0 \}, \quad t_0 \le \tau \le t_1.
    \]
    То есть, алтернированный интеграл Понтрягина \( \mathcal{I}(\tau, t_1, \mathcal{M}) \), сечение моста Красовского 
     \( \mathcal{W}[\tau] \) и множество уровня определенной нами функции \( \mathcal{V}(\tau, x) \) суть одно и то же множество. 
\end{theorem}

Кроме того, функция цены \( \mathcal{V}(\tau, x) \) обладает следующими свойствами.
\begin{theorem}
    1. Справедлив принцип оптимальности
    \[
        \mathcal{V}(\tau, x \mid \mathcal{V}(t_1, \cdot)) = \mathcal{V}(\tau, x \mid \mathcal{V}(t, \cdot \mid \mathcal{V}(t_1,
         \cdot))), \quad \mathcal{V}(t_1, x) \equiv d^2(x, \mathcal{M}), \quad \tau \le t \le t_1. 
    \]
    2. Если существует непрерывная частная производная \( \partial \rho(l \mid \mathcal{W}[t])/ \partial t, \forall l \in
     \mathbb{R}^n \), то фукнция \( \mathcal{V}(t, x) \) является классическим решением уравнения Гамильтона-Якоби-Беллмана-Айзекса.
    \[
        \frac{\partial \mathcal{V}}{\partial t} + \min_u \max_v \left( \frac{\partial \mathcal{V}}{\partial x}, u + v \right) = 0
    \]
    с краевым условием \( \mathcal{V}(t_1, x) = h_+^2(x, \mathcal{M}) \).
\end{theorem}

Если решение \eqref{HJBI}, \eqref{HJBI_boundary} удается найти, то гарантирующая стратегия управления находится следубщим образом:
\begin{equation}
    \mathcal{U}_*(t,x) = \mathrm{argmin} \left\{ \left( \frac{\partial \mathcal{V}}{\partial x}, u \right) : u \in \mathcal{P}(t) \right\},
\end{equation}
если градиент \( \partial \mathcal{V} / \partial x \) существует в точке \( (t, x) \). Или в общем случае
\begin{equation}
    \mathcal{U}_*(t,x) = \left\{ u: \max_v \{ dh_+^2(x, \mathcal{W}[t]) / dt : v \in \mathcal{Q}(t) \} \le 0 \right\}.
\end{equation}

\begin{theorem}\label{dif_incl_theorem}
    Все решения дифференциального включения
    \begin{equation*}
        \dot{x} \in \mathcal{U}_*(t,x) + v(t), \quad x_{\tau} = x(\tau) \in \mathcal{W}[t],
    \end{equation*}
    удовлетворяют включению \( x(t) \in \mathcal{W}[t] , \tau \le t \le t_1 \), и, следовательно, достигают целевого множества \( \mathcal{M} 
     : x(t_1) \in \mathcal{M} \) при любом возмущении \( v(t) \). 
\end{theorem}

Таким образом синтезирующая стратегия \( \mathcal{U}_*(t,x) \) разрешает задачу целевого синтеза управлений при неопределенности.
 Среди допустимых стратегий \( \mathcal{U}(t,x) \), удовлетворяющих включению \( \mathcal{U}(t,x) \subseteq \mathcal{U}_*(t,x) \), то есть
 обеспечивающих выполнение теоремы \eqref{dif_incl_theorem}, находится стратегия "экстремального прицеливания", задаваемая условиями
\begin{equation}
    \mathcal{U}_e(t,x) = \partial_l k(t, -l) = \mathrm{arg min} \{(l^0, u) \mid u \in \mathcal{P}(t) \}.
\end{equation}
Здесь \( \partial_l k \) --- субдифференциал функции \( k(t, l) \) по переменной \( l \), \( k(t,l) = \rho(l \mid
 \mathcal{P}(t)), l^0 = l^0(t, x) \ne 0 \) --- максимизатор задачи 
\begin{equation}
    h_+(x, \mathcal{W}[\tau]) = \max \{(l,x) - \rho(l \mid \mathcal{W}[\tau]) \mid (l,l) \le 1 \},
\end{equation}
причем \( l^0(t,x) = 0 \), если \( h_+(x, \mathcal{W}[\tau)) = 0 \).

Как видно для поиска гарантирующей стратегии достаточно знать лишь сечения \( \mathcal{W}[t] \) функции цены.
 \( \mathcal{V}(t,x) \). Поскольку интегрирование уравнения Гамильтона-Якоби, как правило, очень нетривиально, 
 будем использовать аппроксимации сечения функции цены вместо самой функции. Таким образом сведем задачу к задаче
 поиска аппроксимаций альтернированного интеграла Понтрягина.